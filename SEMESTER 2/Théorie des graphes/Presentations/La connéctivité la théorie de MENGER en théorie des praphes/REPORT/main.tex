%%% PLANTILLA DISEÑADA PARA LA REALIZACIÓN DE DOCUMENTO PARA PROYECTOS INTEGRADORES DE LA FUNDACIÓN UNIVERSITARIA CAFAM - UNICAFAM

%% AUTOR DE LA PLANTILLA: PROFESOR MILLER ALEXANDER QUIROGA CAMPOS
% INGENIERO ELECTRÓNICO & MsC. MATEMÁTICAS APLICADA
% FACULTADA DE INGENIERÍA, TECNOLOGÍA Y CIENCIAS BÁSICAS
%\documentclass[12pt, letterpaper]{article}

\documentclass[12pt,a4paper,oneside]{report}
\usepackage[spanish]{babel}
\usepackage{blindtext}
\usepackage[T1]{fontenc}
\usepackage{times}
\usepackage[utf8]{inputenc}
\usepackage{amsmath}
\usepackage{graphicx}
\usepackage{multicol}
\usepackage{longtable}
\usepackage[refpages]{gloss}
\usepackage{float}
\usepackage{anysize}
\usepackage{appendix}
\usepackage{lscape} 
\usepackage{pdflscape}
\usepackage{multirow}
\usepackage{listings}
\usepackage{color}
\usepackage{setspace}
\usepackage{enumerate} 
\usepackage{upgreek}
\usepackage{cmll}
\usepackage{hyperref}
\usepackage{graphicx} %LaTeX package to import graphics
\graphicspath{{imagenes/}} %configuring the graphicx package






\usepackage{float} % Required for the [H] option




\begin{document}



%----------------------------------------------------------------------------------------
%	CONFIGURACION
%----------------------------------------------------------------------------------------

\marginsize{3.0cm}{3.0cm}{4.0cm}{3.0cm}
\renewcommand*{\contentsname}{Table des Matières}
\renewcommand*{\listtablename}{Liste des tableaux}
\renewcommand*{\listfigurename}{Liste des figures}
\renewcommand{\baselinestretch}{1.0}
\renewcommand{\appendixname}{Annexes}
\renewcommand{\appendixtocname}{Annexes}
\renewcommand{\appendixpagename}{Annexes}
\renewcommand{\thetable}{\arabic{chapter}.\arabic{table}}
\renewcommand*{\tablename}{Tableau}
\renewcommand*{\chaptername}{Chapitre}
\renewcommand*{\thechapter}{\Roman{chapter}}
\renewcommand{\thesection}{\arabic{chapter}.\arabic{section}}
\renewcommand{\figurename}{Figure}
\renewcommand{\thefigure}{\arabic{chapter}.\arabic{figure}}
\renewcommand{\theequation}{\arabic{chapter}.\arabic{equation}}


%----------------------------------------------------------------------------------------
%	PORTADA
%----------------------------------------------------------------------------------------

\begin{titlepage}
 
\begin{center}
 
 {\huge \bf Université Ibn Zohr }\\
\vspace{1cm}
{\Large Faculté Polydisciplinaire - Ouarzazate}\\ 



\begin{center}
    \includegraphics[width=0.6\textwidth]{Assets/fpo_logo.png}
\end{center}

\vspace{1cm}
{\bf \large Module : Théorie des graphes}\\[2.5cm] 

{\bf \large LA CONNÉCTIVITÉ LA THÉORIE DE MENGER EN THÉORIE DES GPRAPHES}\\[1cm]

{\bf \large Master Mathématiques Appliquées pour la Science des Données}\\[0.5cm] 

\vspace{1cm}

\textbf{Élèves :}\\
\textsc{GAJJA Noureddin}  \\
\textsc{BOUHLALI Abdelfattah}  %Nom des élèves

\vspace{1cm}

\emph{\textbf{Enseignant :}\\  
\textsc{Mourad El Ouali} } %Nom de l'enseignant

\vspace{1cm}
\date{\today}
{\large OUARZAZATE ,    03-05-2024}\\[0.2cm] 
\end{center}

\end{titlepage}

\newpage
\newpage
%----------------------------------------------------------------------------------------
%	TABLA DE CONTENIDOS
%---------------------------------------------------------------------------------------

\tableofcontents
\cleardoublepage
\makegloss
\newpage
%----------------------------------------------------------------------------------------
%	INTRODUCTION 
%----------------------------------------------------------------------------------------
\chapter{Concepts Introductifs}
\begin{frame}
\frametitle{Introduction au Théorème de Menger}
\begin{tcolorbox}[colback=orange!10,colframe=orange!100!black,
    title= Introduction]
    Le théorème de Menger est un principe central en théorie des graphes qui décrit la relation entre la connectivité et les chemins indépendants.
\end{tcolorbox}
\end{frame}



\begin{frame}
\frametitle{Théorème de Menger pour les arêtes}
\begin{tcolorbox}[colback=orange!10,colframe=orange!100!black,
    title=La Connectivité et les Chemins Indépendants: Une Perspective des Arêtes]
    Le théorème de Menger pour les arêtes énonce que pour deux sommets non adjacents \( u \) et \( v \) dans un graphe non orienté, le nombre minimal d'arêtes à supprimer pour déconnecter \( u \) de \( v \), noté \( \lambda(u, v) \), est égal au nombre maximal de chemins indépendants les reliant, noté \( \kappa(u, v) \). Formellement, nous avons :
    $$ \lambda(u, v) = \kappa(u, v) $$
    où \( \lambda(u, v) \) est la connectivité d'arête et \( \kappa(u, v) \) est le nombre de chemins indépendants entre \( u \) et \( v \).
\end{tcolorbox}
\end{frame}


%----------------------------------------------------------------------------------------
%	MENGER
%----------------------------------------------------------------------------------------
\chapter{Concepts Introductifs}
\begin{frame}
\frametitle{Introduction au Théorème de Menger}
\begin{tcolorbox}[colback=orange!10,colframe=orange!100!black,
    title= Introduction]
    Le théorème de Menger est un principe central en théorie des graphes qui décrit la relation entre la connectivité et les chemins indépendants.
\end{tcolorbox}
\end{frame}



\begin{frame}
\frametitle{Théorème de Menger pour les arêtes}
\begin{tcolorbox}[colback=orange!10,colframe=orange!100!black,
    title=La Connectivité et les Chemins Indépendants: Une Perspective des Arêtes]
    Le théorème de Menger pour les arêtes énonce que pour deux sommets non adjacents \( u \) et \( v \) dans un graphe non orienté, le nombre minimal d'arêtes à supprimer pour déconnecter \( u \) de \( v \), noté \( \lambda(u, v) \), est égal au nombre maximal de chemins indépendants les reliant, noté \( \kappa(u, v) \). Formellement, nous avons :
    $$ \lambda(u, v) = \kappa(u, v) $$
    où \( \lambda(u, v) \) est la connectivité d'arête et \( \kappa(u, v) \) est le nombre de chemins indépendants entre \( u \) et \( v \).
\end{tcolorbox}
\end{frame}


%----------------------------------------------------------------------------------------
%	DISTANCE ET CONNEXITE
%----------------------------------------------------------------------------------------
\chapter{Concepts Introductifs}
\begin{frame}
\frametitle{Introduction au Théorème de Menger}
\begin{tcolorbox}[colback=orange!10,colframe=orange!100!black,
    title= Introduction]
    Le théorème de Menger est un principe central en théorie des graphes qui décrit la relation entre la connectivité et les chemins indépendants.
\end{tcolorbox}
\end{frame}



\begin{frame}
\frametitle{Théorème de Menger pour les arêtes}
\begin{tcolorbox}[colback=orange!10,colframe=orange!100!black,
    title=La Connectivité et les Chemins Indépendants: Une Perspective des Arêtes]
    Le théorème de Menger pour les arêtes énonce que pour deux sommets non adjacents \( u \) et \( v \) dans un graphe non orienté, le nombre minimal d'arêtes à supprimer pour déconnecter \( u \) de \( v \), noté \( \lambda(u, v) \), est égal au nombre maximal de chemins indépendants les reliant, noté \( \kappa(u, v) \). Formellement, nous avons :
    $$ \lambda(u, v) = \kappa(u, v) $$
    où \( \lambda(u, v) \) est la connectivité d'arête et \( \kappa(u, v) \) est le nombre de chemins indépendants entre \( u \) et \( v \).
\end{tcolorbox}
\end{frame}










%----------------------------------------------------------------------------------------
%	BIBLIOGRAFIA
%----------------------------------------------------------------------------------------
\begin{thebibliography}{9}

\bibitem{menger}
K. Menger, 
\textit{Zur allgemeinen Kurventheorie},
Fund. Math. \textbf{10} (1927), 96--115.

\bibitem{graphtheory}
R. Diestel, 
\textit{Graph Theory},
Graduate Texts in Mathematics, Volume 173, Springer-Verlag, Heidelberg, 5th edition, 2017.

\bibitem{networks}
M.E.J. Newman, 
\textit{Networks: An Introduction},
Oxford University Press, Oxford, 2010.

\bibitem{connectivity}
S. Even, 
\textit{Graph Algorithms},
Computer Science Press, 1979.

\end{thebibliography}




\end{document} 

