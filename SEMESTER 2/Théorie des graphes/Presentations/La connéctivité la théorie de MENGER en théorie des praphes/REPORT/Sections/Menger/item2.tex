\section{Théorème de Menger pour les arêtes}

Ce théorème concerne les graphes et énonce les conditions sous lesquelles on peut trouver un maximum de chemins disjoints entre deux nœuds en coupant un minimum d'arêtes. Plus formellement, dans un graphe non orienté et non pondéré, le théorème de Menger affirme que la taille minimale d'un ensemble d'arêtes dont la suppression déconnecte deux nœuds est égale au nombre maximum de chemins disjoints entre ces deux nœuds.