
\section{THÉORÈME DE MENGER (1927)}

\begin{minipage}{\linewidth}
\begin{center}
\begin{minipage}{0.8\textwidth}
Soient \textcolor{blue}{$u$} et \textcolor{red}{$v$} deux sommets non adjacents d’un graphe connexe $G = (V,E)$. La taille minimum d’un sous-ensemble de sommets séparant \textcolor{blue}{$u$} et \textcolor{red}{$v$} est égale au nombre maximum de chemins deux à deux indépendants joignant \textcolor{blue}{$u$} et \textcolor{red}{$v$}.
\end{minipage}
\end{center}
\end{minipage}

