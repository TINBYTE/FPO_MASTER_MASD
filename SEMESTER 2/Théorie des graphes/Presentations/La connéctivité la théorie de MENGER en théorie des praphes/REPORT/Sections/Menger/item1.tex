\section{introductin a Théorème de Menger}


Le Théorème de Menger, nommé d'après l'éminent mathématicien autrichien Karl Menger, est un résultat fondamental en topologie. Ce théorème énonce des conditions pour qu'un ensemble de chemins dans un espace métrique forme une base de ce dernier.

Plus précisément, le Théorème de Menger stipule que dans un espace métrique séparé, un ensemble infini de points peut former une base si et seulement si pour tout ensemble fini de points, il existe un nombre fini de chemins, chaque chemin reliant deux points différents de cet ensemble et aucun point n'étant inclus dans plus de deux chemins.