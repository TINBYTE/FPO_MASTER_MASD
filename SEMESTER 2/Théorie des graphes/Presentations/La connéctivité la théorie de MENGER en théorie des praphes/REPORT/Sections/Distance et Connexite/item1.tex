
\section{Distance entre les Nœuds}
La distance \( d(u, v) \) entre deux nœuds \( u \) et \( v \) dans un graphe est définie comme le nombre minimal d'arêtes dans un chemin le plus court reliant \( u \) à \( v \). Si aucun chemin n'existe entre \( u \) et \( v \), la distance est considérée comme infinie, symbolisée par \( d(u, v) = \infty \). Cette métrique est fondamentale pour l'analyse des propriétés structurelles et fonctionnelles des graphes.
