\begin{frame}
\frametitle{Théorème de Menger pour les arêtes}
\begin{tcolorbox}[colback=orange!10,colframe=orange!100!black,
    title=La Connectivité et les Chemins Indépendants: Une Perspective des Arêtes]
    Le théorème de Menger pour les arêtes énonce que pour deux sommets non adjacents \( u \) et \( v \) dans un graphe non orienté, le nombre minimal d'arêtes à supprimer pour déconnecter \( u \) de \( v \), noté \( \lambda(u, v) \), est égal au nombre maximal de chemins indépendants les reliant, noté \( \kappa(u, v) \). Formellement, nous avons :
    $$ \lambda(u, v) = \kappa(u, v) $$
    où \( \lambda(u, v) \) est la connectivité d'arête et \( \kappa(u, v) \) est le nombre de chemins indépendants entre \( u \) et \( v \).
\end{tcolorbox}
\end{frame}
