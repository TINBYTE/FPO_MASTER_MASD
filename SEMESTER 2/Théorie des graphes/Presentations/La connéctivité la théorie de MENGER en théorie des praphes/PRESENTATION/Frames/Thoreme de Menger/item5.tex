\begin{frame}
\frametitle{Le Pont entre Théorie et Pratique : Le Théorème de Menger}
\begin{tcolorbox}[colback=orange!10,colframe=orange!100!black,
    title=\textbf{Théorème de Menger (1927)}]
    Énoncé par Karl Menger en 1927, le théorème de Menger est un pilier de la théorie des graphes. Il établit un lien fondamental entre la connectivité locale et globale dans les graphes non orientés.

    Formule du Théorème de Menger :
    $$ \kappa(G) = \min_{S \subseteq V} \left\{ |S| : G - S \text{ est non connexe ou } |G-S| \leq 1 \right\} $$
    où \( \kappa(G) \) est la connectivité du graphe \( G \), \( S \) est un ensemble de sommets, et \( G - S \) représente le graphe après la suppression des sommets dans \( S \).
\end{tcolorbox}
\end{frame}
