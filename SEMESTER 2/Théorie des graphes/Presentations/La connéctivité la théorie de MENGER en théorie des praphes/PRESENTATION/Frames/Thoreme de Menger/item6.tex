\begin{frame}
\frametitle{\textbf{Implications et Extensions : Corollaire du Théorème de Menger}}
    \begin{tcolorbox}[colback=orange!10,colframe=orange!100!black,
        title=\textbf{Corollaire du Théorème de Menger (1927)}]
        Le corollaire du théorème de Menger traite des cas spécifiques et des extensions du théorème original, offrant une perspective plus large sur la connectivité des graphes.

        \textbf{Formule du Corollaire :}
        Pour tout graphe \( G \) et deux sommets non adjacents \( u \) et \( v \), le nombre minimum de sommets à supprimer pour que \( u \) et \( v \) soient dans des composantes connexes distinctes est égal au nombre maximum de chemins \( u-v \) indépendants.
        $$ \kappa'(u,v) = \max \{ \text{nombre de chemins \( u-v \) indépendants} \} $$
        où \( \kappa'(u,v) \) représente la connectivité dans le contexte du corollaire.
    \end{tcolorbox}
\end{frame}

\begin{frame}
\frametitle{\textbf{Corollaire du Théorème de Menger : Un Exemple Illustratif}}
    \begin{tcolorbox}[colback=orange!10,colframe=orange!100!black,
        title=\textbf{Application du Corollaire}]
        Prenons un graphe \( G \) avec des sommets \( u \) et \( v \). Supposons que nous avons identifié trois chemins indépendants entre \( u \) et \( v \). Le corollaire du théorème de Menger nous indique que le nombre minimum de sommets à supprimer pour séparer \( u \) de \( v \) est égal à ce nombre de chemins indépendants.

        \textbf{Application du Corollaire :}
        Si \( u \) et \( v \) sont des nœuds dans un réseau de communication, et qu'il y a trois chemins de communication indépendants entre eux, alors :
        $$ \kappa'(u,v) = 3 $$
        Cela signifie que pour interrompre la communication entre \( u \) et \( v \), il faudrait supprimer au moins trois nœuds du réseau.
    \end{tcolorbox}
\end{frame}


