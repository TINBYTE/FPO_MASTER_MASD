\begin{frame}
\frametitle{Exploitation du Théorème de Menger pour la Résolution des Problématiques de Connectivité}
    \begin{tcolorbox}[colback=orange!10,colframe=orange!100!black,
        title=Application du Théorème de Menger]
        Le théorème de Menger est un résultat fondamental en théorie des graphes qui énonce que pour deux sommets non adjacents \( u \) et \( v \) dans un graphe, le nombre minimum de sommets à enlever pour séparer \( u \) de \( v \) est égal au nombre maximum de chemins indépendants de \( u \) à \( v \). Formellement, cela peut être exprimé comme suit :
        $$ \kappa(u,v) = \max \{ \text{nombre de chemins indépendants de } u \text{ à } v \} $$
        où \( \kappa(u,v) \) représente la connectivité locale entre \( u \) et \( v \).
        
        Ce théorème est utilisé pour résoudre des problèmes complexes de connectivité dans les graphes, aidant à concevoir des réseaux plus robustes et efficaces.
    \end{tcolorbox}
\end{frame}
