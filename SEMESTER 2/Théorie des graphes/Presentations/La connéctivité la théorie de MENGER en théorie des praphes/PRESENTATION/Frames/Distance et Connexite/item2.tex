\begin{frame}
\frametitle{Connectivité dans les Graphes : Variétés, Implications et Exemples}
    \begin{tcolorbox}[colback=orange!10,colframe=orange!100!black,
        title=La connectivité dans les graphes]
        La connectivité dans les graphes a des implications variées dans de nombreux domaines, tels que les réseaux informatiques, la planification urbaine et la biologie. Elle est définie comme suit :
        \begin{itemize}
            \item \textbf{Connectivité de sommets ($\kappa$)}: Le nombre minimum de sommets dont la suppression entraîne un graphe non connexe ou réduit le graphe à un seul sommet.
            \item \textbf{Connectivité d'arêtes ($\lambda$)}: Le nombre minimum d'arêtes dont la suppression rend le graphe non connexe.
        \end{itemize}
        Ces deux mesures sont liées par l'inégalité suivante :
        $$ \kappa(G) \leq \lambda(G) \leq \delta(G) $$
        où $\delta(G)$ est le degré minimum d'un sommet dans le graphe $G$.
    \end{tcolorbox}
\end{frame}
