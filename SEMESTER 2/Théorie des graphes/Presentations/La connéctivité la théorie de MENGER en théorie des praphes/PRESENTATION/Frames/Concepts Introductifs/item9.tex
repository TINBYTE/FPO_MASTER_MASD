\begin{frame}
\frametitle{Connectivité dans les Graphes : Concepts et Applications}
\begin{itemize}
    \item La \textbf{connectivité} mesure la robustesse d'un graphe.
    \item Un \textbf{graphe fortement connecté} permet un chemin dans chaque direction entre chaque paire de sommets.
    \item Un \textbf{graphe faiblement connecté} reste connecté même si l'orientation des arêtes est ignorée.
    \item La \textbf{connectivité de sommet} et la \textbf{connectivité d'arête} sont des mesures de la robustesse d'un graphe.
    \item Le \textbf{théorème de Menger} est un résultat fondamental qui relie la connectivité à des chemins indépendants.
\end{itemize}
\end{frame}