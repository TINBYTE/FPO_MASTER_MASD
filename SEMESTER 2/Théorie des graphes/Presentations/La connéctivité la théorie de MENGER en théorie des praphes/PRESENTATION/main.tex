\documentclass[10pt,hyperref={colorlinks,citecolor=blue,urlcolor=peking_blue,linkcolor=}]{beamer}
\usepackage{Bydevmar}
\usefonttheme{serif}
\usepackage{lipsum}
%\usepackage[scheme = plain]{ctex}
\usepackage{hyperref}
\usepackage{charter} % Nicer fonts
% other packages
\usepackage{latexsym,amsmath,xcolor,multicol,booktabs,calligra}
\usepackage{amssymb}
\usepackage{graphicx}
\usepackage{bm}
\usepackage{natbib}
\usepackage{wrapfig}
\usepackage{amsfonts} 
\usepackage{ragged2e}
\usepackage{parskip}

\apptocmd{\frame}{}{\justifying}{} % Allow optional arguments after frame.

\newcommand{\theHalgorithm}{\arabic{algorithm}}
\theoremstyle{plain}
\newtheorem{axiom}{Axiom}
\newtheorem{claim}[axiom]{Claim}
\newtheorem{assumption}{Assumption}
\newtheorem{remark}{Remark}
\newtheorem{proposition}{Proposition}
\setbeamertemplate{theorems}[numbered] 


% change for your title page information
\author[FPO]{Faculté Polydisciplinaire - Ouarzazate}
\title{LA CONNÉCTIVITÉ LA THÉORIE DE MENGER EN THÉORIE DES GPRAPHES}
\subtitle{Module : Théorie des graphes}
\institute{Master Mathématiques Appliquées pour la Science des Données}
\date{
2024/05/04}

\newif\ifplacelogo % create a new conditional
\placelogotrue % set it to true
\logo{
\ifplacelogo
\includegraphics[width=2cm]{../Figures/PKU-China-logo.png}
\fi
}
% official colors match with the PKU red
\def\cmd#1{\texttt{\color{red}\footnotesize $\backslash$#1}}
\def\env#1{\texttt{\color{blue}\footnotesize #1}}
\definecolor{deepblue}{rgb}{0,0,0.5}
\definecolor{deepred}{rgb}{0.6,0,0}
\definecolor{deepgreen}{rgb}{0,0.5,0}
\definecolor{halfgray}{gray}{0.55}


\usepackage{tcolorbox}

% Définition de la couleur orange
\definecolor{orange}{RGB}{255,140,0}
\begin{document}
{
\setbeamertemplate{logo}{}
\begin{frame}
    \titlepage
    \begin{figure}[htpb]
        \begin{center}
            \includegraphics[width=0.7\linewidth]{Figures/fpo_logo.png}
        \end{center}
    \end{figure}
\end{frame}
}

\placelogofalse

%%%%%%%%%%%
% Concepts Introductifs
%%%%%%%%%%%
\chapter{Concepts Introductifs}
\begin{frame}
\frametitle{Introduction au Théorème de Menger}
\begin{tcolorbox}[colback=orange!10,colframe=orange!100!black,
    title= Introduction]
    Le théorème de Menger est un principe central en théorie des graphes qui décrit la relation entre la connectivité et les chemins indépendants.
\end{tcolorbox}
\end{frame}



\begin{frame}
\frametitle{Théorème de Menger pour les arêtes}
\begin{tcolorbox}[colback=orange!10,colframe=orange!100!black,
    title=La Connectivité et les Chemins Indépendants: Une Perspective des Arêtes]
    Le théorème de Menger pour les arêtes énonce que pour deux sommets non adjacents \( u \) et \( v \) dans un graphe non orienté, le nombre minimal d'arêtes à supprimer pour déconnecter \( u \) de \( v \), noté \( \lambda(u, v) \), est égal au nombre maximal de chemins indépendants les reliant, noté \( \kappa(u, v) \). Formellement, nous avons :
    $$ \lambda(u, v) = \kappa(u, v) $$
    où \( \lambda(u, v) \) est la connectivité d'arête et \( \kappa(u, v) \) est le nombre de chemins indépendants entre \( u \) et \( v \).
\end{tcolorbox}
\end{frame}

%%%%%%%%%%%%%%%%%%%%%%%%%%%%%%%


%%%%%%%%%%%
% Theoreme De Menger
%%%%%%%%%%%
\chapter{Concepts Introductifs}
\begin{frame}
\frametitle{Introduction au Théorème de Menger}
\begin{tcolorbox}[colback=orange!10,colframe=orange!100!black,
    title= Introduction]
    Le théorème de Menger est un principe central en théorie des graphes qui décrit la relation entre la connectivité et les chemins indépendants.
\end{tcolorbox}
\end{frame}



\begin{frame}
\frametitle{Théorème de Menger pour les arêtes}
\begin{tcolorbox}[colback=orange!10,colframe=orange!100!black,
    title=La Connectivité et les Chemins Indépendants: Une Perspective des Arêtes]
    Le théorème de Menger pour les arêtes énonce que pour deux sommets non adjacents \( u \) et \( v \) dans un graphe non orienté, le nombre minimal d'arêtes à supprimer pour déconnecter \( u \) de \( v \), noté \( \lambda(u, v) \), est égal au nombre maximal de chemins indépendants les reliant, noté \( \kappa(u, v) \). Formellement, nous avons :
    $$ \lambda(u, v) = \kappa(u, v) $$
    où \( \lambda(u, v) \) est la connectivité d'arête et \( \kappa(u, v) \) est le nombre de chemins indépendants entre \( u \) et \( v \).
\end{tcolorbox}
\end{frame}

%%%%%%%%%%%%%%%%%%%%%%%%%%%%%%%


%%%%%%%%%%%
% Distance et Connexite
%%%%%%%%%%%
\chapter{Concepts Introductifs}
\begin{frame}
\frametitle{Introduction au Théorème de Menger}
\begin{tcolorbox}[colback=orange!10,colframe=orange!100!black,
    title= Introduction]
    Le théorème de Menger est un principe central en théorie des graphes qui décrit la relation entre la connectivité et les chemins indépendants.
\end{tcolorbox}
\end{frame}



\begin{frame}
\frametitle{Théorème de Menger pour les arêtes}
\begin{tcolorbox}[colback=orange!10,colframe=orange!100!black,
    title=La Connectivité et les Chemins Indépendants: Une Perspective des Arêtes]
    Le théorème de Menger pour les arêtes énonce que pour deux sommets non adjacents \( u \) et \( v \) dans un graphe non orienté, le nombre minimal d'arêtes à supprimer pour déconnecter \( u \) de \( v \), noté \( \lambda(u, v) \), est égal au nombre maximal de chemins indépendants les reliant, noté \( \kappa(u, v) \). Formellement, nous avons :
    $$ \lambda(u, v) = \kappa(u, v) $$
    où \( \lambda(u, v) \) est la connectivité d'arête et \( \kappa(u, v) \) est le nombre de chemins indépendants entre \( u \) et \( v \).
\end{tcolorbox}
\end{frame}

%%%%%%%%%%%%%%%%%%%%%%%%%%%%%%%


%%%%%%%%%%%
% Distance et Connexite
%%%%%%%%%%%
\section{Conclusion}

\begin{frame}{Récapitulation des points clés}
    \begin{itemize}
        \item Les \textbf{accords de Bâle} et la \textbf{loi Dodd-Frank} sont des réglementations essentielles pour la stabilité financière mondiale.
        \item Les Accords de Bâle visent à renforcer la sécurité des banques et à réduire les risques systémiques.
        \item La Loi Dodd-Frank a introduit des réformes importantes pour réguler le secteur financier et protéger les consommateurs.
    \end{itemize}
\end{frame}

\begin{frame}{Importance continue des réglementations}
    \begin{itemize}
        \item Les défis financiers évoluent constamment (nouveaux produits, technologies, crises).
        \item Les accords de Bâle et la loi Dodd-Frank doivent s'adapter pour maintenir la stabilité financière.
        \item La coopération internationale reste essentielle pour une régulation efficace.
    \end{itemize}
\end{frame}
%%%%%%%%%%%%%%%%%%%%%%%%%%%%%%%
\end{document}