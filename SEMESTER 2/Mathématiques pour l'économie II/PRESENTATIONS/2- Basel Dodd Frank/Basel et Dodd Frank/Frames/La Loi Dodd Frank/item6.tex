\begin{frame}{La loi Dodd-Frank et son impact sur la réglementation Q}
   La loi Dodd-Frank a abrogé le règlement Q, autorisant les banques à payer des intérêts sur les dépôts à vue pour la première fois depuis plus de 70 ans. Ce changement a été perçu comme une évolution positive par beaucoup, car il offre aux consommateurs davantage d’options pour épargner et gagner des intérêts. Cependant, certains craignaient que l'abrogation du règlement Q puisse également conduire à une prise de risque accrue de la part des banques, dans la mesure où elles seraient désormais en mesure d'offrir des taux de dépôt plus élevés pour attirer les clients. 
\end{frame}