\begin{frame}{ Le rôle de la réglementation Q dans la crise financière de 2008}


L’un des principaux facteurs contribuant à la crise financière de 2008 a été le manque de réglementation du secteur bancaire. Plus précisément, le règlement Q, introduit en 1933 dans le cadre de la loi Glass-Steagall, a joué un rôle important dans la crise. Le règlement Q a plafonné le montant des intérêts que les banques pouvaient payer sur certains types de dépôts, ce qui a entraîné un certain nombre de conséquences inattendues. Dans cette section, nous explorerons le rôle de la réglementation Q dans la crise financière de 2008 et examinerons si elle devrait être rétablie ou modifiée à l'avenir.
   
\end{frame}